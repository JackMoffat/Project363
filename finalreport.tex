% Created 2019-12-20 Fri 15:22
% Intended LaTeX compiler: pdflatex
\documentclass{article}
\usepackage[utf8]{inputenc}
\usepackage[T1]{fontenc}
\usepackage{graphicx}
\usepackage{grffile}
\usepackage{longtable}
\usepackage{wrapfig}
\usepackage{rotating}
\usepackage[normalem]{ulem}
\usepackage{amsmath}
\usepackage{textcomp}
\usepackage{amssymb}
\usepackage{capt-of}
\usepackage{hyperref}
\usepackage[english]{babel}
\bibliographystyle{plain}
\author{Jackm}
\date{\today}
\title{The N-back Test}
\hypersetup{
 pdfauthor={Jackm},
 pdftitle={The N-back Test},
 pdfkeywords={},
 pdfsubject={},
 pdfcreator={Emacs 27.0.50 (Org mode 9.1.9)}, 
 pdflang={English}}
\begin{document}

\maketitle
\tableofcontents


\section{Introduction}
\label{sec:orgb4e2f17}

In 1958, Wayne Kirchner invented the n-back test \cite{COULACOGLOU201791}. The n-back test is a visuospatial task that has been shown to improve working memory and attentional skills \cite{COLOM2013712}. The basic mechanisms of the test involve the presentation of continuous stimulis in terms of letters or pictures – for every stimulus presented, the participant has to indicate whether it matches a stimulus that was presented n stimuli ago \cite{Kane}. There are different types of n-back tests known as loads: 3-back test, 2-back test and 1-back test \cite{Forns}.

\section{Hypothesis}
\label{sec:orgef36647}

Our hypothesis was that participants would have a more challenging time remembering things initially which would be reflected in a longer reaction time to congruent stimulis in the 2-back test compared to the reaction time of a 1-back test. However, as n-back tests are shown to improve working and short term memory \cite{LEONDOMINGUEZ2015167}, we expect participants to get better at remembering, reflected in shorter reaction times in responding to congruent stimulis.

\section{Materials/Methods}
\label{sec:org42d605a}
\subsection{Information from Pelegrina et. Al (2015)}
\label{sec:org7c984c0}
From \cite{NormativeData}
\subsubsection{{\bfseries\sffamily TODO} R code: ensure table aligns with a page upon generation in pdf (or insert sideways)}
\label{sec:org245a0a8}
\begin{verbatim}
table4 <- read.csv("./dataFromPaper/csvfpsyg-06-01544.csv")


\end{verbatim}

\begin{center}
\begin{tabular}{llrrrrrrrrrrrrrrr}
 &  & nil & Age group &  &  &  &  &  &  & Age group &  &  &  &  &  & \\
 &  & nil & 7 & 8 & 9 & 10 & 11 & 12 & 13 & 7 & 8 & 9 & 10 & 11 & 12 & 13\\
 &  & nil & ( n = 193) & ( n = 285) & ( n = 310) & ( n = 297) & ( n = 315) & ( n = 253) & ( n = 233) & ( n = 194) & ( n = 307) & ( n = 296) & ( n = 321) & ( n = 286) & ( n = 223) & ( n = 209)\\
Total & M & nil & 14.45 & 16.71 & 17.63 & 20.23 & 21.94 & 22.5 & 24.82 & 14.25 & 15.58 & 18.7 & 21.07 & 23.24 & 24.7 & 27.2\\
 & SD & nil & 8.49 & 9.2 & 9.17 & 9.44 & 9.9 & 9.66 & 9.85 & 8.58 & 8.41 & 9.29 & 9.1 & 9.28 & 9.21 & 8.74\\
 & Percentile & 5 & 3 & 3 & 4 & 4 & 4 & 4 & 4 & 2 & 3 & 4 & 5 & 5 & 7 & 11\\
 &  & 25 & 7 & 10 & 10 & 13 & 14 & 14 & 18 & 8 & 10 & 12 & 13 & 16 & 18 & 23\\
 &  & 50 & 13 & 15 & 16 & 21 & 24 & 24 & 28 & 13 & 15 & 18 & 22 & 25 & 26 & 30\\
 &  & 75 & 22 & 25 & 25 & 28 & 30 & 30 & 32 & 20 & 22 & 26 & 28 & 31 & 31 & 34\\
 &  & 95 & 28 & 32 & 32 & 35 & 35 & 37 & 37 & 30 & 30 & 34 & 35 & 35 & 38 & 39\\
1-back & M & nil & 8.05 & 8.82 & 9.14 & 9.72 & 10.01 & 10.2 & 10.28 & 8.21 & 8.6 & 9.54 & 10.11 & 10.41 & 10.73 & 11.29\\
 & SD & nil & 3.04 & 3.09 & 2.82 & 2.96 & 2.98 & 2.96 & 2.96 & 3.29 & 3.09 & 3.12 & 2.86 & 2.77 & 2.55 & 2.34\\
 & Percentile & 5 & 3 & 3 & 4 & 4 & 4 & 4 & 4 & 2 & 3 & 4 & 5 & 5 & 6 & 7\\
 &  & 25 & 6 & 7 & 7 & 8 & 8 & 8 & 9 & 6 & 6 & 7 & 8 & 9 & 9 & 10\\
 &  & 50 & 8 & 9 & 9 & 10 & 11 & 11 & 11 & 8 & 9 & 10 & 11 & 11 & 11 & 12\\
 &  & 75 & 10 & 11 & 11 & 12 & 12 & 13 & 12 & 11 & 11 & 12 & 12 & 12 & 13 & 13\\
 &  & 95 & 13 & 13 & 13 & 14 & 14 & 14 & 14 & 14 & 13 & 14 & 14 & 14 & 14 & 14\\
2-back & M & nil & 4.1 & 5.02 & 5.23 & 6.29 & 6.96 & 7.26 & 8.16 & 3.96 & 4.53 & 5.74 & 6.52 & 7.44 & 7.93 & 9.11\\
 & SD & nil & 3.64 & 3.67 & 3.81 & 3.78 & 4.08 & 3.91 & 3.99 & 3.59 & 3.49 & 3.76 & 3.5 & 3.68 & 3.77 & 3.75\\
 & Percentile & 5 & – & – & – & – & – & – & – & – & – & – & – & – & – & 2\\
 &  & 25 & – & 2 & 2 & 3 & 4 & 5 & 6 & – & 1 & 3 & 4 & 5 & 6 & 7\\
 &  & 50 & 4 & 5 & 5 & 6 & 7 & 8 & 9 & 3 & 5 & 6 & 7 & 8 & 8 & 10\\
 &  & 75 & 7 & 8 & 8 & 9 & 10 & 11 & 11 & 6 & 7 & 9 & 9 & 10 & 11 & 12\\
 &  & 95 & 11 & 11 & 12 & 12 & 13 & 13 & 13 & 11 & 10 & 12 & 12 & 13 & 13 & 14\\
3-back & M & nil & 2.3 & 2.87 & 3.26 & 4.23 & 4.97 & 5.04 & 6.37 & 2.07 & 2.45 & 3.41 & 4.44 & 5.38 & 6.05 & 6.8\\
 & SD & nil & 3.41 & 3.81 & 3.91 & 4.15 & 4.27 & 4.12 & 4.27 & 3.46 & 3.46 & 3.98 & 4.24 & 4.28 & 4.36 & 4.11\\
 & Percentile & 5 & – & – & – & – & – & – & – & – & – & – & – & – & – & –\\
 &  & 25 & – & – & – & – & – & – & 2 & – & – & – & – & – & 1 & 5\\
 &  & 50 & – & – & – & 4 & 6 & 6 & 7 & – & – & – & 5 & 6 & 7 & 8\\
 &  & 75 & 5 & 6 & 7 & 8 & 9 & 8 & 10 & 4 & 5 & 7 & 8 & 9 & 10 & 10\\
 &  & 95 & 9 & 10 & 11 & 11 & 11 & 11 & 12 & 10 & 10 & 10 & 11 & 12 & 12 & 13\\
\end{tabular}
\end{center}


\subsubsection{Python Code For}
\label{sec:orgd304c5c}
\begin{verbatim}
import pandas as pd
t4pd = pd.read_csv("./dataFromPaper/csvfpsyg-06-01544.csv")
for i in ['Boys','Girls']:
    print(t4pd[i])
\end{verbatim}

\subsection{Inline usage}
\label{sec:org34c5585}

\section{Results}
\label{sec:org39114a2}
\subsection{Table}
\label{sec:org56c4427}
\subsection{Simple summary statistics}
\label{sec:org6bbef6a}
\subsection{2 plots}
\label{sec:org5e1c110}

\section{Discussion}
\label{sec:org772ad5f}

\section{Bibliography}
\label{sec:orgba8bb68}
need to add the fpsyg-06-01544 citation!
\bibliography{references}

\section{Appendix}
\label{sec:orge57888a}
\subsection{Python Code for n-back test}
\label{sec:org7ddda4a}
\begin{verbatim}
from psychopy import visual, event, core
import pandas as pd
import random
import time as systime

#########
# setup #
#############################

#############
# Make lists / define functions #
#############


def makeMatches(in_list, trials=5,
                threshold=0, n_back=2,
                keep_list_stats=True, verbose=False):
    '''Creates the matches in a given list.if a random number is greater than threshold,
    then match the letters at positions [idx] and [idx-n_back]
    in_list: list of letters, strings, etc
    trials: how many trials to run
    threshold: type(float) in range(0,1)ld
    keep_stats: Bool: will output a list with information on
    the matches (position, character) and their frequency
    verbose: Bool: prints information about the lists for immediate viewing
    '''

    # done this way to avoid changing original list, confirm necessity?
    out_list = [i for i in in_list]
    list_stats = []  # list holding the character and positions it was matched at
    num_matches = 0
    for idx, char in enumerate(in_list):
        if idx > 1:
            if (random.random() > threshold):
                out_list[idx] = in_list[idx-n_back]
                list_stats.append([(idx, idx-2), char]
                                  ) if keep_list_stats else None
                num_matches += 1

                real_match_rate = num_matches / (len(in_list) - 2)
                # show _stats or not
                if verbose:  # switch this out of a print statement for final thing so it doesnt show up
                    print(
                        f"{num_matches} of {len(in_list)-2} possible matches: {real_match_rate* 100} %")
                    print(f"in_list\n", in_list, "\nmatched list\n", out_list)
                else:
                    pass

                if keep_list_stats:
                    list_stats.insert(0, [(num_matches), "number of matches"])
                    list_stats.insert(0, [(real_match_rate), "actual match rate"])
        return(out_list, list_stats)
    else:
        return(out_list)


#####################
# create trial list #
#####################

n_trials = 15
# need to think of this inverted with how the code is currently written
match_frequency_threshold = 0.5
alphabet = [i for i in "ABCDEFGHIJKLMNOPQRSTUVWXYZ"]
initial_letters = [random.choice(alphabet) for i in range(n_trials)]

trial_list = makeMatches(initial_letters, trials=n_trials,
                         threshold=match_frequency_threshold, keep_list_stats=False)
ptt = 1.2
# ptt is the amount of time between trials, stands for "per time trial"

######################
# Window setup below #
######################
mywin = visual.Window(fullscr=True, screen=0, allowGUI=False, allowStencil=False,
                      monitor='testMonitor', color=[0, 0, 0], colorSpace='rgb')

clock = core.Clock()  # this is a clock

press_times = []  # List records the data


##############################

intro = True

if intro:
    # TODO  Find out how to display the last sentence in text_string
    text_string = f"This is an N-Back task.  This task is a test of working memory.  You will be presented with a random series of letters, one by one.  For this task, you will press the spacebar if you see a letter that was repeated two letters back.  For example, if you see a sequence such as A, D, A, then you will have to press the spacebar.  You will be given a sequence of {n_trials} letters.  "
    textList = text_string.split("  ")
    for msg in textList:
        displayMsg = visual.TextStim(
            mywin, text=msg, pos=(0.5, 0))
        mywin.flip()
        displayMsg.draw()
        core.wait(3.5)

    countdownMessage = visual.TextStim(
        mywin, text='The task will begin after this countdown.', pos=(0.5, 0))
    countdownMessage.autoDraw = True
    mywin.flip()
    core.wait(3.5)
    countdownMessage.text = ' '
    mywin.flip()
    core.wait(0.5)



countdownString = "5,4,3,2,1"
countdown = countdownString.split(',')
# ct is the countdown timer

for num in countdown:
    txtDisplay = visual.TextStim(
        mywin, text = num , alignHoriz='left', alignVert='center', pos=(0, 0))
    mywin.flip()
    txtDisplay.draw()
    core.wait(1.0)


###################
# display letters #
###################

trialTime = core.Clock()

for idx, char in enumerate(trial_list):

    trialLength = core.CountdownTimer()
    keys = event.getKeys(keyList=["space"], timeStamped = trialLength)
    txtDisplay.text = char
    mywin.flip()
    txtDisplay.draw()
    print(keys, trialLength.getTime(), txtDisplay.text)
    press_times.append([keys, trialLength.getTime(), txtDisplay.text])
    core.wait(ptt)
    txtDisplay.text = "+"
    mywin.flip()
    txtDisplay.draw()
    core.wait(ptt)
    trialLength.reset()
    # currently appending in tuple form list_stats = []  # list holding the character and positions it was matched at

endMessage = visual.TextStim(
    mywin, text = ' ', pos=(0.5, 0))
endMessage.autoDraw=True
mywin.flip()
core.wait(1.5)
endMessage.text = 'You have completed the N-Back task. Thank you!'
mywin.flip()
core.wait(3.0)

print(press_times)

ts = systime.localtime()
timestamp = str(systime.strftime("Y%yM%mD%dH%HM%MS%S",ts))
datafile = open(f"datafile_{timestamp}.txt", "w+")

################
# writing file #
################
for line in press_times:
    datafile.write(str(line))
    datafile.write("\n")
    datafile.close()

# #not sure needed
# for line in n_list:
#     datafile.write(line,)
#     datafile.write("\n")

# for line in stats:
#     datafile.write(line)
#     datafile.write("\n")

\end{verbatim}
\subsection{Data from Our Python Code}
\label{sec:org8170de3}
\end{document}
